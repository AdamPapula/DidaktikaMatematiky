\documentclass[12pt,a4paper,addpoints]{exam}

\pagestyle{empty}

% Packages
\usepackage{../Styles/defpackages}
\usepackage{../Styles/mathcmd}
\usepackage{../Styles/optionalcmd}
\usepackage{pgfplots}
\pgfplotsset{compat=1.15}
\usepackage{mathrsfs}

% Assets
\def\hmargin{18mm}
\def\vmargin{25mm}
\newcommand{\rightnote}[1]{\hspace*{\fill} $\triangleleft$ \textit{#1}}

\geometry{
	top=\vmargin,
	bottom=\vmargin,
	left=\hmargin,
	right=\hmargin
}
\setlength{\parindent}{0pt}
\setlength{\parskip}{\baselineskip}
\setlength{\dottedlinefillheight}{1cm}


% Title page info
\def\maintitle{Didaktika informatiky}
\def\subtitle{Zadání a~řešení písemné práce -- výroky a~množiny}
\def\authorname{Kateřina Novotná, Adam Papula}

% Language adjustment
\chpword{b.}
\pointpoints{b.}{b.}
\bonuspointpoints{b. -- \textbf{bonusová úloha}}{b. -- \textbf{bonusová úloha}}

%Checkbox adjustment
\checkboxchar{$\Box$}
\checkedchar{$\blacksquare$}

\begin{document}
    \begin{titlepage}
        \begin{center}
            \Large\textbf{{\maintitle}}

            \normalsize
            \vspace{0.5cm}
                \subtitle
            \vspace{1.5cm}
            
            \textbf{\authorname}
            \vspace{1.5cm}

            \today
            \vfill
            
            \raggedright
                \textbf{Čas:} 15-20 minut\\
                \textbf{Cíle testu:}
                \noindent
                \footnotesize
                \begin{itemize}[topsep=0pt]
                \item Úloha č. \ref{question@1}\\
                        \todo{Bloom}
                \item Úloha č. \ref{question@2}\\
                        \todo{Bloom}
                \item Úloha č. \ref{question@3}\\
                        \todo{Bloom}
                \item Úloha č. \ref{question@4}\\
                        \todo{Bloom}
                \item Úloha č. \ref{question@5}\\
                        \todo{Bloom}
                \item Úloha č. \ref{question@6}\\
                        \todo{Bloom}
                \item Úloha č. \ref{question@7}\\
                        \todo{Bloom}

                \end{itemize}
        \end{center}
    \end{titlepage}

    % \printanswers
    \begin{center}
\ifprintanswers
    \large \textbf{Písemná práce:} výroky a množiny \textbf{(varianta A)} -- řešení
\else
    \large \textbf{Písemná práce:} výroky a množiny \textbf{(varianta A)}
    
    \normalsize
    \makebox[8cm]{\textsc{Jméno:}\enspace\hrulefill}\qquad
    \makebox[3cm]{\textsc{Třída:}\enspace\hrulefill}\qquad
    \makebox[4cm]{\textsc{Datum:}\enspace\hrulefill}
    \end{center}

    \begin{table}[h]
        \scriptsize
        \centering
        \begin{tabular}{rccccc}
            \toprule
            \textbf{Body} & $15$ & $14-12$ & $11-9$ & $8-6$ & $5-0$ \\ 
            \textbf{Známka} & $1$ & $2$ & $3$ & $4$ & $5$ \\ 
            \bottomrule
        \end{tabular}
    \end{table}
\fi


\begin{questions}
    \question
        Určete, zda se jedná o výroky:
        \begin{parts}
            \part[\half]\TFQuestion{T}{Číslo 12 je prvočíslo.}
            \part[\half]\TFQuestion{F}{Přines mi prosím kapesník.}
            \part[\half]\TFQuestion{T}{$\forall x \in \mathbb{Z}: x + 3 > 0$}
        \end{parts}
    
    \question
        Určete negace kvantifikovaných výroků:   
        \begin{parts}
            \part[1]Alespoň jeden cestující nevystoupil.
            \begin{solutionordottedlines}[1cm]
                Všichni cestující vystoupili.
            \end{solutionordottedlines}

            \part[1]Právě jedna moje učebnice je těžká.
            \begin{solutionordottedlines}[1cm]
                Žádná moje učebnice nebo alespoň dvě moje učebnice jsou těžké.
            \end{solutionordottedlines}
        \end{parts}

    \question
        Negujte následující výroky:
        \begin{parts}
            \part[2]Každé přirozené číslo, které je dělitelné dvaceti, je dělitelné čtyřmi. 
            \begin{solutionordottedlines}[2cm]
                Existuje alespoň jedno přirozené číslo, které je dělitelné dvaceti a~není dělitelné čtyřmi.
            \end{solutionordottedlines}

            \part[2]Do kina půjdu s Terkou nebo s Eliškou.
            \begin{solutionordottedlines}[1cm]
                Do kina nepůjdu s Terkou a nepůjdu ani s Eliškou.
            \end{solutionordottedlines}
        \end{parts}

    \question[1 \half]
        Z~následujících dvou \textit{symbolicky zapsaných kvantifikovaných výroků} si vyberte \textbf{jeden}, který vyjádříte slovy. Dále rozhodněte o~jeho pravdivosti:
        \begin{parts}
            \part $\forall x \in \mathbb{R}: \sqrt{x^2} = |x|$
            \part $\exists x \in \mathbb{R} \forall y \in \mathbb{R}: x \cdot y = y$
            \begin{solutionordottedlines}[2cm]
                \begin{itemize}
                    \item[(a)]  Výrok je pravdivý.\\
                                Druhá odmocnina z druhé mocniny libovolného reálného čísla je rovna jeho absolutní hodnotě. 
                    \item[(b)]  Výrok je pravdivý.\\
                                Existuje takové reálné číslo $x$, že pro všechna reálná čísla $y$ platí $x \cdot y = y$  
                \end{itemize}
            \end{solutionordottedlines}
        \end{parts}

    \question[1]
        Vypište všechny prvky následující množiny:
        $$ M = \{\xi \in \mathbb{Z}: -27 < \xi^3 \leq 8 \} $$
        \begin{solution}[2cm]
            $$M = \{ -2, -1, 0, 1, 2\}$$ 
        \end{solution}
    
    \question[3] 
        Mějme zadány intervaly $A = \langle 0, 18 \rangle$, $B = (13, 28) $ a $C = \langle 15, 17 \rangle$. Určete $((A \displaystyle \cap B) \setminus C)'$
        \begin{solution}[5cm]
            \begin{align*}
                A \displaystyle \cap B                  &= ( 13,18 \rangle\\
                (A \displaystyle \cap B) \setminus C    &= (13,15) \displaystyle \cup (17,18\rangle\\
                ((A \displaystyle \cap B) \setminus C)' &= (-\infty, 13\rangle \displaystyle \cup  \langle 15,17\rangle \displaystyle \cup (18, \infty)
            \end{align*}
        \end{solution}

    \question[2]
        Ve třídě je 29~žáků, 19~z~nich umí lyžovat, 12~jezdí na snowboardu, 5~jich nelyžuje a~ani nejezdí na snowboardu. 
        Znázorněte pomocí Vennova diagramu a~určete, kolik žáků umí lyžovat i~jezdit na snowboardu.

        \begin{solution}[10cm]
            Označme si množinu všech žáků třídy jako $T$, $|T| = 29$. Žáky, kteří umí lyžovat označíme $L$, $|L| = 19$. 
            Snowboardisty označíme $S$, $|S| = 12$. Žáků, kteří neumí ani lyžovat ano na snowboardu je celkem $5$. Tedy 
                $$ |L \displaystyle \cup S| = 29 - 5 = 24 $$
            žáků umí buď lyžovat, nebo na snowboardu nebo obojí. Nyní, pokud sečteme žáky, co umí lyžovat a na snowboardu
            dostaneme 
                $$ 19 + 12 = 31\text{,} $$
            což odpovídá případu, kdy neexistuje ani jeden žák co umí na lyžích a~snowboardu zároveň. Jelikož ale platí $31 > 24$,
            dostaneme informaci, že celkem 
                $$ 31 - 24 = \doubleunderline{7} $$
            žáků umí na lyžích a snowboardu zároveň.

            $T$\\
            \begin{venndiagram2sets}[labelA={$S$},labelB={$L$},labelOnlyA={5},labelOnlyB={12},
                labelAB={7},labelNotAB={5}]
                \setkeys{venn}{shade=pink!50!white}
                \fillACapB
            \end{venndiagram2sets}
        \end{solution}        
\end{questions}



    \begin{center}
\ifprintanswers
    \large \textbf{Písemná práce:} výroky a množiny \textbf{(varianta B)} -- řešení
\else
    \large \textbf{Písemná práce:} výroky a množiny \textbf{(varianta B)}

    \normalsize
    \makebox[8cm]{\textsc{Jméno:}\enspace\hrulefill}\qquad
    \makebox[3cm]{\textsc{Třída:}\enspace\hrulefill}\qquad
    \makebox[4cm]{\textsc{Datum:}\enspace\hrulefill}
    \end{center}

    \begin{table}[h]
        \scriptsize
        \centering
        \begin{tabular}{rccccc}
            \toprule
            \textbf{Body} & $15$ & $14-12$ & $11-9$ & $8-6$ & $5-0$ \\ 
            \textbf{Známka} & $1$ & $2$ & $3$ & $4$ & $5$ \\ 
            \bottomrule
        \end{tabular}
    \end{table}
\fi

\begin{questions}
    \question 
        Určete, zda se jedná o výroky:
        \begin{parts}
            \part[\half]\TFQuestion{T}{$\exists! x \in \mathbb{R}, \forall y \in \mathbb{N}: \frac{y}{x} = 0$}
            \part[\half]\TFQuestion{T}{Všechna prvočísla jsou lichá.}
            \part[\half]\TFQuestion{F}{Berlín patří mezi racionální čísla.}
        \end{parts}
        
    \vspace{1cm}

    \question
        Určete negace kvantifikovaných výroků:   
        \begin{parts}
            \part[1]Všichni moji kamarádi mají hnědé oči
            \begin{solutionordottedlines}[1cm]
                Alespoň jeden můj kamarád nemá hnědé oči.
            \end{solutionordottedlines}
            \vspace{1cm}
            \part[1]Alespoň 4 dny v týdnu bude pršet.
            \begin{solutionordottedlines}[1cm]
                Nejvýše 3 dny v týdnu bude pršet.
            \end{solutionordottedlines}
        \end{parts}
        
    \vspace{1cm}

    \question
        Negujte následující výroky:
        \begin{parts}
            \part[2]Existuje alespoň jeden trojúhelník, ve kterém se všechny jeho výšky neprotínají v jediném bodě. 
            \begin{solutionordottedlines}[2cm]
                V každém trojúhelníku se všechny jeho výšky protínají v jediném bodě.
            \end{solutionordottedlines}
            \vspace{1cm}
            \part[2]Sní polévku právě tehdy, když v ní nebude zelenina. 
            \begin{solutionordottedlines}[1cm]
                Sní polévku a bude v ní zelenina nebo v polévce zelenina nebude a polévku nesní.
            \end{solutionordottedlines}
        \end{parts}
        
    \newpage

    \question[1 \half]
        Z~následujících dvou \textit{symbolicky zapsaných kvantifikovaných výroků} si vyberte \textbf{alespoň jeden} (druhý je bonusový), který vyjádříte slovy. Dále rozhodněte o~jeho pravdivosti:
        \begin{parts}
            \part $\forall x \in \mathbb{R}: x^2 > 0$
            \part $\forall a \in \mathbb{R}, \forall b \in \mathbb{R}: a = b \iff a^2 = b^2$
            \begin{solutionordottedlines}[2cm]
                \begin{itemize}
                    \item[(a)]  Výrok není pravdivý.\\
                                Druhá mocnina každého reálného čísla je větší než nula. 
                    \item[(b)]  Výrok není pravdivý.\\
                                Pro každá dvě reálná čísla platí: čísla se sobě rovnají právě tehdy, když se rovnají jejich druhé mocniny.  
                \end{itemize}
            \end{solutionordottedlines}
        \end{parts}
        
    \vspace{1cm}

    \question[1]
        Vypište všechny prvky následující množiny:
        $$ S = \{\chi \in \mathbb{N}: -18 < \chi^3 \leq 64 | \chi \text{ je sudé}\} $$
        \begin{solution}[2cm]
            $$S = \{2,4\}$$ 
        \end{solution}
    
    \question[3] 
        Mějme zadány intervaly $A = (-5, 12)$, $B = \langle4, 11) $ a $C = \langle 3, 5 \rangle$.\\
        Určete $((A \setminus B) \cup C)'$
        \begin{solution}[5cm]
            \begin{align*}
                A  \setminus B                  &= (-5,4) \cup \langle 11,12)\\
                (A  \setminus B) \cup C         &= (-5,5\rangle \cup \langle11,12)\\
                ((A  \cap B) \setminus C)'      &= (-\infty, -5\rangle  \cup  (-5,11)  \cup \langle 12, \infty)
            \end{align*}
        \end{solution}

    \question[2]
        Ve třídě hraje 21 žáků fotbal nebo košíkovou, 4 žáci z této třídy nehrají ani fotbal, ani košíkovou, 12 žáků hraje 
        košíkovou, 14 žáků hraje fotbal. Znázorněte pomocí Vennova diagramu a~určete, kolik žáků hraje pouze fotbal.
        
        \begin{solution}[6cm]
            Označme si množinu všech žáků třídy jako $T$. Žáky, kteří hrají fotbal označíme $F$ a žáky hrající košíkovou $K$. 
            Víme, že 
                $$ |F \cap K| = 21\text{.} $$
            Dále ze zadání víme, že 
                $$ |(F \cap K)'| = 4\text{.} $$
            Pro množinu $K$ platí $|K| = 12$ a pro množinu $F$ platí $|F| = 14$. Máme tedy
                $$12 + 14 = 26\text{,}$$
            přičemž sportuje pouze $21$ žáků, tedy
                $$26-21 = 5$$
            žáků hraje obě hry. Nyní nám stačí dopočítat počet žáků hrající pouze fotbal,
                $$14 - 5 = \doubleunderline{9}$$ 

            $T$\\
            \begin{venndiagram2sets}[labelA={$F$},labelB={$K$},labelOnlyA={9},labelOnlyB={7},
                labelAB={5},labelNotAB={4}]
                \setkeys{venn}{shade=pink!50!white}
                \fillOnlyA
            \end{venndiagram2sets}
        \end{solution}        
\end{questions}




    \printanswers
    \begin{center}
\ifprintanswers
    \large \textbf{Písemná práce:} výroky a množiny \textbf{(varianta A)} -- řešení
\else
    \large \textbf{Písemná práce:} výroky a množiny \textbf{(varianta A)}
    
    \normalsize
    \makebox[8cm]{\textsc{Jméno:}\enspace\hrulefill}\qquad
    \makebox[3cm]{\textsc{Třída:}\enspace\hrulefill}\qquad
    \makebox[4cm]{\textsc{Datum:}\enspace\hrulefill}
    \end{center}

    \begin{table}[h]
        \scriptsize
        \centering
        \begin{tabular}{rccccc}
            \toprule
            \textbf{Body} & $15$ & $14-12$ & $11-9$ & $8-6$ & $5-0$ \\ 
            \textbf{Známka} & $1$ & $2$ & $3$ & $4$ & $5$ \\ 
            \bottomrule
        \end{tabular}
    \end{table}
\fi


\begin{questions}
    \question
        Určete, zda se jedná o výroky:
        \begin{parts}
            \part[\half]\TFQuestion{T}{Číslo 12 je prvočíslo.}
            \part[\half]\TFQuestion{F}{Přines mi prosím kapesník.}
            \part[\half]\TFQuestion{T}{$\forall x \in \mathbb{Z}: x + 3 > 0$}
        \end{parts}
    
    \question
        Určete negace kvantifikovaných výroků:   
        \begin{parts}
            \part[1]Alespoň jeden cestující nevystoupil.
            \begin{solutionordottedlines}[1cm]
                Všichni cestující vystoupili.
            \end{solutionordottedlines}

            \part[1]Právě jedna moje učebnice je těžká.
            \begin{solutionordottedlines}[1cm]
                Žádná moje učebnice nebo alespoň dvě moje učebnice jsou těžké.
            \end{solutionordottedlines}
        \end{parts}

    \question
        Negujte následující výroky:
        \begin{parts}
            \part[2]Každé přirozené číslo, které je dělitelné dvaceti, je dělitelné čtyřmi. 
            \begin{solutionordottedlines}[2cm]
                Existuje alespoň jedno přirozené číslo, které je dělitelné dvaceti a~není dělitelné čtyřmi.
            \end{solutionordottedlines}

            \part[2]Do kina půjdu s Terkou nebo s Eliškou.
            \begin{solutionordottedlines}[1cm]
                Do kina nepůjdu s Terkou a nepůjdu ani s Eliškou.
            \end{solutionordottedlines}
        \end{parts}

    \question[1 \half]
        Z~následujících dvou \textit{symbolicky zapsaných kvantifikovaných výroků} si vyberte \textbf{jeden}, který vyjádříte slovy. Dále rozhodněte o~jeho pravdivosti:
        \begin{parts}
            \part $\forall x \in \mathbb{R}: \sqrt{x^2} = |x|$
            \part $\exists x \in \mathbb{R} \forall y \in \mathbb{R}: x \cdot y = y$
            \begin{solutionordottedlines}[2cm]
                \begin{itemize}
                    \item[(a)]  Výrok je pravdivý.\\
                                Druhá odmocnina z druhé mocniny libovolného reálného čísla je rovna jeho absolutní hodnotě. 
                    \item[(b)]  Výrok je pravdivý.\\
                                Existuje takové reálné číslo $x$, že pro všechna reálná čísla $y$ platí $x \cdot y = y$  
                \end{itemize}
            \end{solutionordottedlines}
        \end{parts}

    \question[1]
        Vypište všechny prvky následující množiny:
        $$ M = \{\xi \in \mathbb{Z}: -27 < \xi^3 \leq 8 \} $$
        \begin{solution}[2cm]
            $$M = \{ -2, -1, 0, 1, 2\}$$ 
        \end{solution}
    
    \question[3] 
        Mějme zadány intervaly $A = \langle 0, 18 \rangle$, $B = (13, 28) $ a $C = \langle 15, 17 \rangle$. Určete $((A \displaystyle \cap B) \setminus C)'$
        \begin{solution}[5cm]
            \begin{align*}
                A \displaystyle \cap B                  &= ( 13,18 \rangle\\
                (A \displaystyle \cap B) \setminus C    &= (13,15) \displaystyle \cup (17,18\rangle\\
                ((A \displaystyle \cap B) \setminus C)' &= (-\infty, 13\rangle \displaystyle \cup  \langle 15,17\rangle \displaystyle \cup (18, \infty)
            \end{align*}
        \end{solution}

    \question[2]
        Ve třídě je 29~žáků, 19~z~nich umí lyžovat, 12~jezdí na snowboardu, 5~jich nelyžuje a~ani nejezdí na snowboardu. 
        Znázorněte pomocí Vennova diagramu a~určete, kolik žáků umí lyžovat i~jezdit na snowboardu.

        \begin{solution}[10cm]
            Označme si množinu všech žáků třídy jako $T$, $|T| = 29$. Žáky, kteří umí lyžovat označíme $L$, $|L| = 19$. 
            Snowboardisty označíme $S$, $|S| = 12$. Žáků, kteří neumí ani lyžovat ano na snowboardu je celkem $5$. Tedy 
                $$ |L \displaystyle \cup S| = 29 - 5 = 24 $$
            žáků umí buď lyžovat, nebo na snowboardu nebo obojí. Nyní, pokud sečteme žáky, co umí lyžovat a na snowboardu
            dostaneme 
                $$ 19 + 12 = 31\text{,} $$
            což odpovídá případu, kdy neexistuje ani jeden žák co umí na lyžích a~snowboardu zároveň. Jelikož ale platí $31 > 24$,
            dostaneme informaci, že celkem 
                $$ 31 - 24 = \doubleunderline{7} $$
            žáků umí na lyžích a snowboardu zároveň.

            $T$\\
            \begin{venndiagram2sets}[labelA={$S$},labelB={$L$},labelOnlyA={5},labelOnlyB={12},
                labelAB={7},labelNotAB={5}]
                \setkeys{venn}{shade=pink!50!white}
                \fillACapB
            \end{venndiagram2sets}
        \end{solution}        
\end{questions}



    \begin{center}
\ifprintanswers
    \large \textbf{Písemná práce:} výroky a množiny \textbf{(varianta B)} -- řešení
\else
    \large \textbf{Písemná práce:} výroky a množiny \textbf{(varianta B)}

    \normalsize
    \makebox[8cm]{\textsc{Jméno:}\enspace\hrulefill}\qquad
    \makebox[3cm]{\textsc{Třída:}\enspace\hrulefill}\qquad
    \makebox[4cm]{\textsc{Datum:}\enspace\hrulefill}
    \end{center}

    \begin{table}[h]
        \scriptsize
        \centering
        \begin{tabular}{rccccc}
            \toprule
            \textbf{Body} & $15$ & $14-12$ & $11-9$ & $8-6$ & $5-0$ \\ 
            \textbf{Známka} & $1$ & $2$ & $3$ & $4$ & $5$ \\ 
            \bottomrule
        \end{tabular}
    \end{table}
\fi

\begin{questions}
    \question 
        Určete, zda se jedná o výroky:
        \begin{parts}
            \part[\half]\TFQuestion{T}{$\exists! x \in \mathbb{R}, \forall y \in \mathbb{N}: \frac{y}{x} = 0$}
            \part[\half]\TFQuestion{T}{Všechna prvočísla jsou lichá.}
            \part[\half]\TFQuestion{F}{Berlín patří mezi racionální čísla.}
        \end{parts}
        
    \vspace{1cm}

    \question
        Určete negace kvantifikovaných výroků:   
        \begin{parts}
            \part[1]Všichni moji kamarádi mají hnědé oči
            \begin{solutionordottedlines}[1cm]
                Alespoň jeden můj kamarád nemá hnědé oči.
            \end{solutionordottedlines}
            \vspace{1cm}
            \part[1]Alespoň 4 dny v týdnu bude pršet.
            \begin{solutionordottedlines}[1cm]
                Nejvýše 3 dny v týdnu bude pršet.
            \end{solutionordottedlines}
        \end{parts}
        
    \vspace{1cm}

    \question
        Negujte následující výroky:
        \begin{parts}
            \part[2]Existuje alespoň jeden trojúhelník, ve kterém se všechny jeho výšky neprotínají v jediném bodě. 
            \begin{solutionordottedlines}[2cm]
                V každém trojúhelníku se všechny jeho výšky protínají v jediném bodě.
            \end{solutionordottedlines}
            \vspace{1cm}
            \part[2]Sní polévku právě tehdy, když v ní nebude zelenina. 
            \begin{solutionordottedlines}[1cm]
                Sní polévku a bude v ní zelenina nebo v polévce zelenina nebude a polévku nesní.
            \end{solutionordottedlines}
        \end{parts}
        
    \newpage

    \question[1 \half]
        Z~následujících dvou \textit{symbolicky zapsaných kvantifikovaných výroků} si vyberte \textbf{alespoň jeden} (druhý je bonusový), který vyjádříte slovy. Dále rozhodněte o~jeho pravdivosti:
        \begin{parts}
            \part $\forall x \in \mathbb{R}: x^2 > 0$
            \part $\forall a \in \mathbb{R}, \forall b \in \mathbb{R}: a = b \iff a^2 = b^2$
            \begin{solutionordottedlines}[2cm]
                \begin{itemize}
                    \item[(a)]  Výrok není pravdivý.\\
                                Druhá mocnina každého reálného čísla je větší než nula. 
                    \item[(b)]  Výrok není pravdivý.\\
                                Pro každá dvě reálná čísla platí: čísla se sobě rovnají právě tehdy, když se rovnají jejich druhé mocniny.  
                \end{itemize}
            \end{solutionordottedlines}
        \end{parts}
        
    \vspace{1cm}

    \question[1]
        Vypište všechny prvky následující množiny:
        $$ S = \{\chi \in \mathbb{N}: -18 < \chi^3 \leq 64 | \chi \text{ je sudé}\} $$
        \begin{solution}[2cm]
            $$S = \{2,4\}$$ 
        \end{solution}
    
    \question[3] 
        Mějme zadány intervaly $A = (-5, 12)$, $B = \langle4, 11) $ a $C = \langle 3, 5 \rangle$.\\
        Určete $((A \setminus B) \cup C)'$
        \begin{solution}[5cm]
            \begin{align*}
                A  \setminus B                  &= (-5,4) \cup \langle 11,12)\\
                (A  \setminus B) \cup C         &= (-5,5\rangle \cup \langle11,12)\\
                ((A  \cap B) \setminus C)'      &= (-\infty, -5\rangle  \cup  (-5,11)  \cup \langle 12, \infty)
            \end{align*}
        \end{solution}

    \question[2]
        Ve třídě hraje 21 žáků fotbal nebo košíkovou, 4 žáci z této třídy nehrají ani fotbal, ani košíkovou, 12 žáků hraje 
        košíkovou, 14 žáků hraje fotbal. Znázorněte pomocí Vennova diagramu a~určete, kolik žáků hraje pouze fotbal.
        
        \begin{solution}[6cm]
            Označme si množinu všech žáků třídy jako $T$. Žáky, kteří hrají fotbal označíme $F$ a žáky hrající košíkovou $K$. 
            Víme, že 
                $$ |F \cap K| = 21\text{.} $$
            Dále ze zadání víme, že 
                $$ |(F \cap K)'| = 4\text{.} $$
            Pro množinu $K$ platí $|K| = 12$ a pro množinu $F$ platí $|F| = 14$. Máme tedy
                $$12 + 14 = 26\text{,}$$
            přičemž sportuje pouze $21$ žáků, tedy
                $$26-21 = 5$$
            žáků hraje obě hry. Nyní nám stačí dopočítat počet žáků hrající pouze fotbal,
                $$14 - 5 = \doubleunderline{9}$$ 

            $T$\\
            \begin{venndiagram2sets}[labelA={$F$},labelB={$K$},labelOnlyA={9},labelOnlyB={7},
                labelAB={5},labelNotAB={4}]
                \setkeys{venn}{shade=pink!50!white}
                \fillOnlyA
            \end{venndiagram2sets}
        \end{solution}        
\end{questions}



\end{document}
