\documentclass[12pt,a4paper,addpoints]{exam}

\pagestyle{empty}

% Packages
\usepackage{../Styles/defpackages}
\usepackage{../Styles/mathcmd}
\usepackage{../Styles/optionalcmd}
\usepackage{pgfplots}
\pgfplotsset{compat=1.15}
\usepackage{mathrsfs}

% Assets
\def\hmargin{18mm}
\def\vmargin{25mm}
\newcommand{\rightnote}[1]{\hspace*{\fill} $\triangleleft$ \textit{#1}}

\geometry{
	top=\vmargin,
	bottom=\vmargin,
	left=\hmargin,
	right=\hmargin
}
\setlength{\parindent}{0pt}
\setlength{\parskip}{\baselineskip}
\setlength{\dottedlinefillheight}{1cm}


% Title page info
\def\maintitle{Didaktika informatiky}
\def\subtitle{Zadání a~řešení písemné práce -- výroky a~množiny}
\def\authorname{Kateřina Novotná, Adam Papula}

% Language adjustment
\chpword{b.}
\pointpoints{b.}{b.}
\bonuspointpoints{b. -- \textbf{bonusová úloha}}{b. -- \textbf{bonusová úloha}}

%Checkbox adjustment
\checkboxchar{$\Box$}
\checkedchar{$\blacksquare$}

\begin{document}
    \begin{titlepage}
        \begin{center}
            \Large\textbf{{\maintitle}}

            \normalsize
            \vspace{0.5cm}
                \subtitle
            \vspace{1.5cm}
            
            \textbf{\authorname}
            \vspace{1.5cm}

            \today
            \vfill
            
            \raggedright
                \textbf{Čas:} 15-20 minut\\
                \textbf{Cíle testu:}
                \noindent
                \footnotesize
                \begin{itemize}[topsep=0pt]
                \item Úloha č. \ref{question@1}\\
                        \todo{Bloom}
                \item Úloha č. \ref{question@2}\\
                        \todo{Bloom}
                \item Úloha č. \ref{question@3}\\
                        \todo{Bloom}
                \item Úloha č. \ref{question@4}\\
                        \todo{Bloom}
                \item Úloha č. \ref{question@5}\\
                        \todo{Bloom}
                \item Úloha č. \ref{question@6}\\
                        \todo{Bloom}
                \item Úloha č. \ref{question@7}\\
                        \todo{Bloom}

                \end{itemize}
        \end{center}
    \end{titlepage}

    % \printanswers
    \begin{center}
\large \textbf{Písemná práce:} exponenciální funkce \textbf{(varianta A)}

\normalsize
\makebox[8cm]{\textsc{Jméno:}\enspace\hrulefill}\qquad
\makebox[3cm]{\textsc{Třída:}\enspace\hrulefill}\qquad
\makebox[4cm]{\textsc{Datum:}\enspace\hrulefill}
\end{center}
\begin{table}[h]
\centering
\begin{tabular}{c|c|c|c|c|c}
    \textbf{Body}   & $10-9$ & $8-7$ & $6-5$ & $4-3$ & $2-0$ \\ \hline
    \textbf{Známka} & $1$     & $2$   & $3 $  & $4$   & $5$
\end{tabular}
\end{table}

\begin{questions}
    \question[3] Řešte v $\R$ rovnici $$4^{\frac{1-x}{1+x}}=\left(\frac{1}{2}\right)^{\frac{1}{3}}$$
    \question[3] Řešte v $\Z$ rovnici $$\sqrt[4]{4^x}\cdot \sqrt[3]{2^{x-3}}=\sqrt[6]{16}$$.
    \question[4] Řešte v $\Z$ rovnici $$\sqrt[5]{3^{8x^2}}\cdot\left(\frac{1}{9}\right)^{\frac{x^2}{2}}\cdot 3^x=\sqrt[5]{27^4}$$.
\end{questions}

\newpage

\begin{center}
\Large\textbf{Vzorové řešení}\normalsize
\begin{enumerate}
    \item Určení podmínek řešitelnosti: $x\neq -1$.\rightnote{1 bod}
    \begin{align*}
        \begin{aligned}
            4^{\frac{1-x}{1+x}}&=\left(\frac{1}{2}\right)^{\frac{1}{3}} & \\
            2^{\frac{2-2x}{1+x}}&=2^{-\frac{1}{3}} & \text{\rightnote{převedení na společný základ (1 bod)}}\\
            \frac{2-2x}{1+x}&=-\frac{1}{3} & \\
            6-6x&=-1-x & \\
            7&=5x & \\
            x&=\frac{7}{5} & \text{\rightnote{správný výsledek (1 bod)}}
        \end{aligned}
    \end{align*}
    \item Definiční obor rovnice není omezen.
    \begin{align*}
        \begin{aligned}
            \sqrt[4]{4^x}\cdot \sqrt[3]{2^{x-3}}&=\sqrt[6]{16} & \\
            2^{\frac{x}{2}}\cdot 2^{\frac{x-3}{3}}&=2^{\frac{2}{3}} & \text{\rightnote{převedení na společný základ (2 body)}}\\
            \frac{x}{2}+\frac{x-3}{3}&=\frac{2}{3} & \\
            3x+2x-6&=4 & \\
            5x&=10 & \\
            x&=2 & \text{\rightnote{správný výsledek (1 bod)}}
        \end{aligned}
    \end{align*}
    \item Definiční obor rovnice není omezen.
    \begin{align*}
        \begin{aligned}
            \sqrt[5]{3^{8x^2}}\cdot\left(\frac{1}{9}\right)^{\frac{x^2}{2}}\cdot 3^x&=\sqrt[5]{27^4} & \\
            3^{\frac{8}{5}}\cdot 3^{-x^2}\cdot 3^x&=3^{\frac{12}{5}}& \text{\rightnote{převedení na společný základ (2 body)}}\\
            \frac{8}{5}x^2-x^2+x&=\frac{12}{5} & \\
            3x^2+5x-12&=0 & \\
            x_{1,2}&=\frac{-5\pm\sqrt{5^2-4\cdot 3\cdot (-12)}}{2\cdot 3} & \\
            x_1=-3 &\land \cancel{x_2=-\frac{4}{3}} & \text{\rightnote{správný výsledek a vyřazení $x_2$ (2 body)}} \\
        \end{aligned}
    \end{align*}
\end{enumerate}
\end{center}

    \begin{center}
\ifprintanswers
    \large \textbf{Písemná práce:} výroky a množiny \textbf{(varianta B)} -- řešení
\else
    \large \textbf{Písemná práce:} výroky a množiny \textbf{(varianta B)}

    \normalsize
    \makebox[8cm]{\textsc{Jméno:}\enspace\hrulefill}\qquad
    \makebox[3cm]{\textsc{Třída:}\enspace\hrulefill}\qquad
    \makebox[4cm]{\textsc{Datum:}\enspace\hrulefill}
    \end{center}

    \begin{table}[h]
        \scriptsize
        \centering
        \begin{tabular}{rccccc}
            \toprule
            \textbf{Body} & $15$ & $14-12$ & $11-9$ & $8-6$ & $5-0$ \\ 
            \textbf{Známka} & $1$ & $2$ & $3$ & $4$ & $5$ \\ 
            \bottomrule
        \end{tabular}
    \end{table}
\fi

\begin{questions}
    \question 
        Určete, zda se jedná o výroky:
        \begin{parts}
            \part[\half]\TFQuestion{T}{$\exists! x \in \mathbb{R}, \forall y \in \mathbb{N}: \frac{y}{x} = 0$}
            \part[\half]\TFQuestion{T}{Všechna prvočísla jsou lichá.}
            \part[\half]\TFQuestion{F}{Berlín patří mezi racionální čísla.}
        \end{parts}
    
    \question
        Určete negace kvantifikovaných výroků:   
        \begin{parts}
            \part[1]Všichni moji kamarádi mají hnědé oči
            \begin{solutionordottedlines}[1cm]
                Alespoň jeden můj kamarád nemá hnědé oči.
            \end{solutionordottedlines}

            \part[1]Alespoň 4 dny v týdnu bude pršet.
            \begin{solutionordottedlines}[1cm]
                Nejvýše 3 dny v týdnu bude pršet.
            \end{solutionordottedlines}
        \end{parts}

    \question
        Negujte následující výroky:
        \begin{parts}
            \part[2]Existuje alespoň jeden trojúhelník, ve kterém se všechny jeho výšky neprotínají v jediném bodě. 
            \begin{solutionordottedlines}[2cm]
                V každém trojúhelníku se všechny jeho výšky protínají v jediném bodě.
            \end{solutionordottedlines}

            \part[2]Sní polévku právě tehdy, když v ní nebude zelenina. 
            \begin{solutionordottedlines}[1cm]
                Sní polévku a bude v ní zelenina nebo v polévce zelenina nebude a polévku nesní.
            \end{solutionordottedlines}
        \end{parts}

    \question[1 \half]
        Z~následujících dvou \textit{symbolicky zapsaných kvantifikovaných výroků} si vyberte \textbf{jeden}, který vyjádříte slovy. Dále rozhodněte o~jeho pravdivosti:
        \begin{parts}
            \part $\forall x \in \mathbb{R}: x^2 > 0$
            \part $\forall a \in \mathbb{R}, \forall b \in \mathbb{R}: a = b \iff a^2 = b^2$
            \begin{solutionordottedlines}[2cm]
                \begin{itemize}
                    \item[(a)]  Výrok není pravdivý.\\
                                Druhá mocnina každého reálného čísla je větší než nula. 
                    \item[(b)]  Výrok není pravdivý.\\
                                Pro každá dvě reálná čísla platí: čísla se sobě rovnají právě tehdy, když se rovnají jejich druhé mocniny.  
                \end{itemize}
            \end{solutionordottedlines}
        \end{parts}

    \question[1]
        Vypište všechny prvky následující množiny:
        $$ S = \{\chi \in \mathbb{N}: -18 < \chi^3 \leq ^64 | \chi \text{ je sudé}\} $$
        \begin{solution}[2cm]
            $$S = \{2,4\}$$ 
        \end{solution}
    
    \question[3] 
        Mějme zadány intervaly $A = (-5, 12)$, $B = \langle4, 11) $ a $C = \langle 3, 5 \rangle$.\\
        Určete $((A \setminus B) \cup C)'$
        \begin{solution}[5cm]
            \begin{align*}
                A  \setminus B                  &= (-5,4) \cup \langle 11,12)\\
                (A  \setminus B) \cup C         &= (-5,5\rangle \cup \langle11,12)\\
                ((A  \cap B) \setminus C)'      &= (-\infty, -5\rangle  \cup  (-5,11)  \cup \langle 12, \infty)
            \end{align*}
        \end{solution}

    \question[2]
        Ve třídě hraje 21 žáků fotbal nebo košíkovou, 4 žácí z této třídy nehrají ani fotbal, ani košíkovou, 12 žáků hraje 
        košíkovou, 14 žáků hraje fotbal. Znázorněte pomocí Vennova diagramu a~určete, kolik žáků hraje pouze fotbal.
        
        \begin{solution}[10cm]
            Označme si množinu všech žáků třídy jako $T$. Žáky, kteří hrají fotbal označíme $F$ a žáky hrající košíkovou $K$. 
            Víme, že 
                $$ |F \cap K| = 21\text{.} $$
            Dále ze zadání víme, že 
                $$ |(F \cap K)'| = 4\text{.} $$
            Pro množinu $K$ platí $|K| = 12$ a pro množinu $F$ platí $|F| = 14$. Máme tedy
                $$12 + 14 = 26\text{,}$$
            přičemž sportuje pouze $21$ žáků, tedy
                $$26-21 = 5$$
            žáků hraje obě hry. Nyní nám stačí dopočítat počet žáků hrající pouze fotbal,
                $$14 - 5 = \doubleunderline{9}$$ 

            $T$\\
            \begin{venndiagram2sets}[labelA={$F$},labelB={$K$},labelOnlyA={9},labelOnlyB={7},
                labelAB={5},labelNotAB={4}]
                \setkeys{venn}{shade=pink!50!white}
                \fillOnlyA
            \end{venndiagram2sets}
        \end{solution}        
\end{questions}




    \printanswers
    \begin{center}
\large \textbf{Písemná práce:} exponenciální funkce \textbf{(varianta A)}

\normalsize
\makebox[8cm]{\textsc{Jméno:}\enspace\hrulefill}\qquad
\makebox[3cm]{\textsc{Třída:}\enspace\hrulefill}\qquad
\makebox[4cm]{\textsc{Datum:}\enspace\hrulefill}
\end{center}
\begin{table}[h]
\centering
\begin{tabular}{c|c|c|c|c|c}
    \textbf{Body}   & $10-9$ & $8-7$ & $6-5$ & $4-3$ & $2-0$ \\ \hline
    \textbf{Známka} & $1$     & $2$   & $3 $  & $4$   & $5$
\end{tabular}
\end{table}

\begin{questions}
    \question[3] Řešte v $\R$ rovnici $$4^{\frac{1-x}{1+x}}=\left(\frac{1}{2}\right)^{\frac{1}{3}}$$
    \question[3] Řešte v $\Z$ rovnici $$\sqrt[4]{4^x}\cdot \sqrt[3]{2^{x-3}}=\sqrt[6]{16}$$.
    \question[4] Řešte v $\Z$ rovnici $$\sqrt[5]{3^{8x^2}}\cdot\left(\frac{1}{9}\right)^{\frac{x^2}{2}}\cdot 3^x=\sqrt[5]{27^4}$$.
\end{questions}

\newpage

\begin{center}
\Large\textbf{Vzorové řešení}\normalsize
\begin{enumerate}
    \item Určení podmínek řešitelnosti: $x\neq -1$.\rightnote{1 bod}
    \begin{align*}
        \begin{aligned}
            4^{\frac{1-x}{1+x}}&=\left(\frac{1}{2}\right)^{\frac{1}{3}} & \\
            2^{\frac{2-2x}{1+x}}&=2^{-\frac{1}{3}} & \text{\rightnote{převedení na společný základ (1 bod)}}\\
            \frac{2-2x}{1+x}&=-\frac{1}{3} & \\
            6-6x&=-1-x & \\
            7&=5x & \\
            x&=\frac{7}{5} & \text{\rightnote{správný výsledek (1 bod)}}
        \end{aligned}
    \end{align*}
    \item Definiční obor rovnice není omezen.
    \begin{align*}
        \begin{aligned}
            \sqrt[4]{4^x}\cdot \sqrt[3]{2^{x-3}}&=\sqrt[6]{16} & \\
            2^{\frac{x}{2}}\cdot 2^{\frac{x-3}{3}}&=2^{\frac{2}{3}} & \text{\rightnote{převedení na společný základ (2 body)}}\\
            \frac{x}{2}+\frac{x-3}{3}&=\frac{2}{3} & \\
            3x+2x-6&=4 & \\
            5x&=10 & \\
            x&=2 & \text{\rightnote{správný výsledek (1 bod)}}
        \end{aligned}
    \end{align*}
    \item Definiční obor rovnice není omezen.
    \begin{align*}
        \begin{aligned}
            \sqrt[5]{3^{8x^2}}\cdot\left(\frac{1}{9}\right)^{\frac{x^2}{2}}\cdot 3^x&=\sqrt[5]{27^4} & \\
            3^{\frac{8}{5}}\cdot 3^{-x^2}\cdot 3^x&=3^{\frac{12}{5}}& \text{\rightnote{převedení na společný základ (2 body)}}\\
            \frac{8}{5}x^2-x^2+x&=\frac{12}{5} & \\
            3x^2+5x-12&=0 & \\
            x_{1,2}&=\frac{-5\pm\sqrt{5^2-4\cdot 3\cdot (-12)}}{2\cdot 3} & \\
            x_1=-3 &\land \cancel{x_2=-\frac{4}{3}} & \text{\rightnote{správný výsledek a vyřazení $x_2$ (2 body)}} \\
        \end{aligned}
    \end{align*}
\end{enumerate}
\end{center}

    \begin{center}
\ifprintanswers
    \large \textbf{Písemná práce:} výroky a množiny \textbf{(varianta B)} -- řešení
\else
    \large \textbf{Písemná práce:} výroky a množiny \textbf{(varianta B)}

    \normalsize
    \makebox[8cm]{\textsc{Jméno:}\enspace\hrulefill}\qquad
    \makebox[3cm]{\textsc{Třída:}\enspace\hrulefill}\qquad
    \makebox[4cm]{\textsc{Datum:}\enspace\hrulefill}
    \end{center}

    \begin{table}[h]
        \scriptsize
        \centering
        \begin{tabular}{rccccc}
            \toprule
            \textbf{Body} & $15$ & $14-12$ & $11-9$ & $8-6$ & $5-0$ \\ 
            \textbf{Známka} & $1$ & $2$ & $3$ & $4$ & $5$ \\ 
            \bottomrule
        \end{tabular}
    \end{table}
\fi

\begin{questions}
    \question 
        Určete, zda se jedná o výroky:
        \begin{parts}
            \part[\half]\TFQuestion{T}{$\exists! x \in \mathbb{R}, \forall y \in \mathbb{N}: \frac{y}{x} = 0$}
            \part[\half]\TFQuestion{T}{Všechna prvočísla jsou lichá.}
            \part[\half]\TFQuestion{F}{Berlín patří mezi racionální čísla.}
        \end{parts}
    
    \question
        Určete negace kvantifikovaných výroků:   
        \begin{parts}
            \part[1]Všichni moji kamarádi mají hnědé oči
            \begin{solutionordottedlines}[1cm]
                Alespoň jeden můj kamarád nemá hnědé oči.
            \end{solutionordottedlines}

            \part[1]Alespoň 4 dny v týdnu bude pršet.
            \begin{solutionordottedlines}[1cm]
                Nejvýše 3 dny v týdnu bude pršet.
            \end{solutionordottedlines}
        \end{parts}

    \question
        Negujte následující výroky:
        \begin{parts}
            \part[2]Existuje alespoň jeden trojúhelník, ve kterém se všechny jeho výšky neprotínají v jediném bodě. 
            \begin{solutionordottedlines}[2cm]
                V každém trojúhelníku se všechny jeho výšky protínají v jediném bodě.
            \end{solutionordottedlines}

            \part[2]Sní polévku právě tehdy, když v ní nebude zelenina. 
            \begin{solutionordottedlines}[1cm]
                Sní polévku a bude v ní zelenina nebo v polévce zelenina nebude a polévku nesní.
            \end{solutionordottedlines}
        \end{parts}

    \question[1 \half]
        Z~následujících dvou \textit{symbolicky zapsaných kvantifikovaných výroků} si vyberte \textbf{jeden}, který vyjádříte slovy. Dále rozhodněte o~jeho pravdivosti:
        \begin{parts}
            \part $\forall x \in \mathbb{R}: x^2 > 0$
            \part $\forall a \in \mathbb{R}, \forall b \in \mathbb{R}: a = b \iff a^2 = b^2$
            \begin{solutionordottedlines}[2cm]
                \begin{itemize}
                    \item[(a)]  Výrok není pravdivý.\\
                                Druhá mocnina každého reálného čísla je větší než nula. 
                    \item[(b)]  Výrok není pravdivý.\\
                                Pro každá dvě reálná čísla platí: čísla se sobě rovnají právě tehdy, když se rovnají jejich druhé mocniny.  
                \end{itemize}
            \end{solutionordottedlines}
        \end{parts}

    \question[1]
        Vypište všechny prvky následující množiny:
        $$ S = \{\chi \in \mathbb{N}: -18 < \chi^3 \leq ^64 | \chi \text{ je sudé}\} $$
        \begin{solution}[2cm]
            $$S = \{2,4\}$$ 
        \end{solution}
    
    \question[3] 
        Mějme zadány intervaly $A = (-5, 12)$, $B = \langle4, 11) $ a $C = \langle 3, 5 \rangle$.\\
        Určete $((A \setminus B) \cup C)'$
        \begin{solution}[5cm]
            \begin{align*}
                A  \setminus B                  &= (-5,4) \cup \langle 11,12)\\
                (A  \setminus B) \cup C         &= (-5,5\rangle \cup \langle11,12)\\
                ((A  \cap B) \setminus C)'      &= (-\infty, -5\rangle  \cup  (-5,11)  \cup \langle 12, \infty)
            \end{align*}
        \end{solution}

    \question[2]
        Ve třídě hraje 21 žáků fotbal nebo košíkovou, 4 žácí z této třídy nehrají ani fotbal, ani košíkovou, 12 žáků hraje 
        košíkovou, 14 žáků hraje fotbal. Znázorněte pomocí Vennova diagramu a~určete, kolik žáků hraje pouze fotbal.
        
        \begin{solution}[10cm]
            Označme si množinu všech žáků třídy jako $T$. Žáky, kteří hrají fotbal označíme $F$ a žáky hrající košíkovou $K$. 
            Víme, že 
                $$ |F \cap K| = 21\text{.} $$
            Dále ze zadání víme, že 
                $$ |(F \cap K)'| = 4\text{.} $$
            Pro množinu $K$ platí $|K| = 12$ a pro množinu $F$ platí $|F| = 14$. Máme tedy
                $$12 + 14 = 26\text{,}$$
            přičemž sportuje pouze $21$ žáků, tedy
                $$26-21 = 5$$
            žáků hraje obě hry. Nyní nám stačí dopočítat počet žáků hrající pouze fotbal,
                $$14 - 5 = \doubleunderline{9}$$ 

            $T$\\
            \begin{venndiagram2sets}[labelA={$F$},labelB={$K$},labelOnlyA={9},labelOnlyB={7},
                labelAB={5},labelNotAB={4}]
                \setkeys{venn}{shade=pink!50!white}
                \fillOnlyA
            \end{venndiagram2sets}
        \end{solution}        
\end{questions}



\end{document}
