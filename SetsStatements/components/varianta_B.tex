\begin{center}
\ifprintanswers
    \large \textbf{Písemná práce:} výroky a množiny \textbf{(varianta B)} -- řešení
\else
    \large \textbf{Písemná práce:} výroky a množiny \textbf{(varianta B)}

    \normalsize
    \makebox[8cm]{\textsc{Jméno:}\enspace\hrulefill}\qquad
    \makebox[3cm]{\textsc{Třída:}\enspace\hrulefill}\qquad
    \makebox[4cm]{\textsc{Datum:}\enspace\hrulefill}
    \end{center}

    \begin{table}[h]
        \scriptsize
        \centering
        \begin{tabular}{rccccc}
            \toprule
            \textbf{Body} & $15$ & $14-12$ & $11-9$ & $8-6$ & $5-0$ \\ 
            \textbf{Známka} & $1$ & $2$ & $3$ & $4$ & $5$ \\ 
            \bottomrule
        \end{tabular}
    \end{table}
\fi

\begin{questions}
    \question 
        Určete, zda se jedná o výroky:
        \begin{parts}
            \part[\half]\TFQuestion{T}{$\exists! x \in \mathbb{R}, \forall y \in \mathbb{N}: \frac{y}{x} = 0$}
            \part[\half]\TFQuestion{T}{Všechna prvočísla jsou lichá.}
            \part[\half]\TFQuestion{F}{Berlín patří mezi racionální čísla.}
        \end{parts}
    
    \question
        Určete negace kvantifikovaných výroků:   
        \begin{parts}
            \part[1]Všichni moji kamarádi mají hnědé oči
            \begin{solutionordottedlines}[1cm]
                Alespoň jeden můj kamarád nemá hnědé oči.
            \end{solutionordottedlines}

            \part[1]Alespoň 4 dny v týdnu bude pršet.
            \begin{solutionordottedlines}[1cm]
                Nejvýše 3 dny v týdnu bude pršet.
            \end{solutionordottedlines}
        \end{parts}

    \question
        Negujte následující výroky:
        \begin{parts}
            \part[2]Existuje alespoň jeden trojúhelník, ve kterém se všechny jeho výšky neprotínají v jediném bodě. 
            \begin{solutionordottedlines}[2cm]
                V každém trojúhelníku se všechny jeho výšky protínají v jediném bodě.
            \end{solutionordottedlines}

            \part[2]Sní polévku právě tehdy, když v ní nebude zelenina. 
            \begin{solutionordottedlines}[1cm]
                Sní polévku a bude v ní zelenina nebo v polévce zelenina nebude a polévku nesní.
            \end{solutionordottedlines}
        \end{parts}

    \question[1 \half]
        Z~následujících dvou \textit{symbolicky zapsaných kvantifikovaných výroků} si vyberte \textbf{jeden}, který vyjádříte slovy. Dále rozhodněte o~jeho pravdivosti:
        \begin{parts}
            \part $\forall x \in \mathbb{R}: x^2 > 0$
            \part $\forall a \in \mathbb{R}, \forall b \in \mathbb{R}: a = b \iff a^2 = b^2$
            \begin{solutionordottedlines}[2cm]
                \begin{itemize}
                    \item[(a)]  Výrok není pravdivý.\\
                                Druhá mocnina každého reálného čísla je větší než nula. 
                    \item[(b)]  Výrok není pravdivý.\\
                                Pro každá dvě reálná čísla platí: čísla se sobě rovnají právě tehdy, když se rovnají jejich druhé mocniny.  
                \end{itemize}
            \end{solutionordottedlines}
        \end{parts}

    \question[1]
        Vypište všechny prvky následující množiny:
        $$ S = \{\chi \in \mathbb{N}: -18 < \chi^3 \leq ^64 | \chi \text{ je sudé}\} $$
        \begin{solution}[2cm]
            $$S = \{2,4\}$$ 
        \end{solution}
    
    \question[3] 
        Mějme zadány intervaly $A = (-5, 12)$, $B = \langle4, 11) $ a $C = \langle 3, 5 \rangle$.\\
        Určete $((A \setminus B) \cup C)'$
        \begin{solution}[5cm]
            \begin{align*}
                A  \setminus B                  &= (-5,4) \cup \langle 11,12)\\
                (A  \setminus B) \cup C         &= (-5,5\rangle \cup \langle11,12)\\
                ((A  \cap B) \setminus C)'      &= (-\infty, -5\rangle  \cup  (-5,11)  \cup \langle 12, \infty)
            \end{align*}
        \end{solution}

    \question[2]
        Ve třídě hraje 21 žáků fotbal nebo košíkovou, 4 žácí z této třídy nehrají ani fotbal, ani košíkovou, 12 žáků hraje 
        košíkovou, 14 žáků hraje fotbal. Znázorněte pomocí Vennova diagramu a~určete, kolik žáků hraje pouze fotbal.
        
        \begin{solution}[10cm]
            Označme si množinu všech žáků třídy jako $T$. Žáky, kteří hrají fotbal označíme $F$ a žáky hrající košíkovou $K$. 
            Víme, že 
                $$ |F \cap K| = 21\text{.} $$
            Dále ze zadání víme, že 
                $$ |(F \cap K)'| = 4\text{.} $$
            Pro množinu $K$ platí $|K| = 12$ a pro množinu $F$ platí $|F| = 14$. Máme tedy
                $$12 + 14 = 26\text{,}$$
            přičemž sportuje pouze $21$ žáků, tedy
                $$26-21 = 5$$
            žáků hraje obě hry. Nyní nám stačí dopočítat počet žáků hrající pouze fotbal,
                $$14 - 5 = \doubleunderline{9}$$ 

            $T$\\
            \begin{venndiagram2sets}[labelA={$F$},labelB={$K$},labelOnlyA={9},labelOnlyB={7},
                labelAB={5},labelNotAB={4}]
                \setkeys{venn}{shade=pink!50!white}
                \fillOnlyA
            \end{venndiagram2sets}
        \end{solution}        
\end{questions}


