\begin{center}
\ifprintanswers
    \large \textbf{Písemná práce:} výroky a množiny \textbf{(varianta A)} -- řešení
\else
    \large \textbf{Písemná práce:} výroky a množiny \textbf{(varianta A)}
    
    \normalsize
    \makebox[8cm]{\textsc{Jméno:}\enspace\hrulefill}\qquad
    \makebox[3cm]{\textsc{Třída:}\enspace\hrulefill}\qquad
    \makebox[4cm]{\textsc{Datum:}\enspace\hrulefill}
    \end{center}

    \begin{table}[h]
        \scriptsize
        \centering
        \begin{tabular}{rccccc}
            \toprule
            \textbf{Body} & $15$ & $14-12$ & $11-9$ & $8-6$ & $5-0$ \\ 
            \textbf{Známka} & $1$ & $2$ & $3$ & $4$ & $5$ \\ 
            \bottomrule
        \end{tabular}
    \end{table}
\fi


\begin{questions}
    \question
        Určete, zda se jedná o výroky:
        \begin{parts}
            \part[\half]\TFQuestion{T}{Číslo 12 je prvočíslo.}
            \part[\half]\TFQuestion{F}{Přines mi prosím kapesník.}
            \part[\half]\TFQuestion{T}{$\forall x \in \mathbb{Z}: x + 3 > 0$}
        \end{parts}
    
    \question
        Určete negace kvantifikovaných výroků:   
        \begin{parts}
            \part[1]Alespoň jeden cestující nevystoupil.
            \begin{solutionordottedlines}[1cm]
                Všichni cestující vystoupili.
            \end{solutionordottedlines}

            \part[1]Právě jedna moje učebnice je těžká.
            \begin{solutionordottedlines}[1cm]
                Žádná moje učebnice nebo alespoň dvě moje učebnice jsou těžké.
            \end{solutionordottedlines}
        \end{parts}

    \question
        Negujte následující výroky:
        \begin{parts}
            \part[2]Každé přirozené číslo, které je dělitelné dvaceti, je dělitelné čtyřmi. 
            \begin{solutionordottedlines}[2cm]
                Existuje alespoň jedno přirozené číslo, které je dělitelné dvaceti a~není dělitelné čtyřmi.
            \end{solutionordottedlines}

            \part[2]Do kina půjdu s Terkou nebo s Eliškou.
            \begin{solutionordottedlines}[1cm]
                Do kina nepůjdu s Terkou a nepůjdu ani s Eliškou.
            \end{solutionordottedlines}
        \end{parts}

    \question[1 \half]
        Z~následujících dvou \textit{symbolicky zapsaných kvantifikovaných výroků} si vyberte \textbf{jeden}, který vyjádříte slovy. Dále rozhodněte o~jeho pravdivosti:
        \begin{parts}
            \part $\forall x \in \mathbb{R}: \sqrt{x^2} = |x|$
            \part $\exists x \in \mathbb{R} \forall y \in \mathbb{R}: x \cdot y = y$
            \begin{solutionordottedlines}[2cm]
                \begin{itemize}
                    \item[(a)]  Výrok je pravdivý.\\
                                Druhá odmocnina z druhé mocniny libovolného reálného čísla je rovna jeho absolutní hodnotě. 
                    \item[(b)]  Výrok je pravdivý.\\
                                Existuje takové reálné číslo $x$, že pro všechna reálná čísla $y$ platí $x \cdot y = y$  
                \end{itemize}
            \end{solutionordottedlines}
        \end{parts}

    \question[1]
        Vypište všechny prvky následující množiny:
        $$ M = \{\xi \in \mathbb{Z}: -27 < \xi^3 \leq 8 \} $$
        \begin{solution}[2cm]
            $$M = \{ -2, -1, 0, 1, 2\}$$ 
        \end{solution}
    
    \question[3] 
        Mějme zadány intervaly $A = \langle 0, 18 \rangle$, $B = (13, 28) $ a $C = \langle 15, 17 \rangle$. Určete $((A \displaystyle \cap B) \setminus C)'$
        \begin{solution}[5cm]
            \begin{align*}
                A \displaystyle \cap B                  &= ( 13,18 \rangle\\
                (A \displaystyle \cap B) \setminus C    &= (13,15) \displaystyle \cup (17,18\rangle\\
                ((A \displaystyle \cap B) \setminus C)' &= (-\infty, 13\rangle \displaystyle \cup  \langle 15,17\rangle \displaystyle \cup (18, \infty)
            \end{align*}
        \end{solution}

    \question[2]
        Ve třídě je 29~žáků, 19~z~nich umí lyžovat, 12~jezdí na snowboardu, 5~jich nelyžuje a~ani nejezdí na snowboardu. 
        Znázorněte pomocí Vennova diagramu a~určete, kolik žáků umí lyžovat i~jezdit na snowboardu.

        \begin{solution}[10cm]
            Označme si množinu všech žáků třídy jako $T$, $|T| = 29$. Žáky, kteří umí lyžovat označíme $L$, $|L| = 19$. 
            Snowboardisty označíme $S$, $|S| = 12$. Žáků, kteří neumí ani lyžovat ano na snowboardu je celkem $5$. Tedy 
                $$ |L \displaystyle \cup S| = 29 - 5 = 24 $$
            žáků umí buď lyžovat, nebo na snowboardu nebo obojí. Nyní, pokud sečteme žáky, co umí lyžovat a na snowboardu
            dostaneme 
                $$ 19 + 12 = 31\text{,} $$
            což odpovídá případu, kdy neexistuje ani jeden žák co umí na lyžích a~snowboardu zároveň. Jelikož ale platí $31 > 24$,
            dostaneme informaci, že celkem 
                $$ 31 - 24 = \doubleunderline{7} $$
            žáků umí na lyžích a snowboardu zároveň.

            $T$\\
            \begin{venndiagram2sets}[labelA={$S$},labelB={$L$},labelOnlyA={5},labelOnlyB={12},
                labelAB={7},labelNotAB={5}]
                \setkeys{venn}{shade=pink!50!white}
                \fillACapB
            \end{venndiagram2sets}
        \end{solution}        
\end{questions}


