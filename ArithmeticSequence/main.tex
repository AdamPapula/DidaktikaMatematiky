\documentclass[12pt,a4paper,addpoints]{article}

\pagestyle{empty}

% Packages
\usepackage{../Styles/defpackages}
\usepackage{../Styles/mathcmd}
\usepackage{../Styles/optionalcmd}
\usepackage{pgfplots}
\pgfplotsset{compat=1.15}
\usepackage{mathrsfs}

% Assets
\def\hmargin{18mm}
\def\vmargin{25mm}
\newcommand{\rightnote}[1]{\hspace*{\fill} $\triangleleft$ \textit{#1}}

\geometry{
	top=\vmargin,
	bottom=\vmargin,
	left=\hmargin,
	right=\hmargin
}
\setlength{\parindent}{0pt}
\setlength{\parskip}{\baselineskip}
\setlength{\dottedlinefillheight}{1cm}


% Title page info
\def\maintitle{Didaktika matematiky}
\def\subtitle{Motivace a zavedení konkrétního pojmu školské matematiky}
\def\authorname{Kateřina Novotná, Adam Papula}


\begin{document}
    \begin{titlepage}
        \begin{center}
            \Large\textbf{{\maintitle}}

            \normalsize
            \vspace{0.5cm}
                \subtitle
            \vspace{1.5cm}
            
            \textbf{\authorname}
            \vspace{1.5cm}

            \today
            \vfill
            
            \raggedright
                \textbf{Téma:} aritmetická posloupnost -- diference; vzorce pro $n$-tý člen, součet prvních $n$ členů.
                
                \textbf{Cílová skupina žáků:} 3. ročník SŠ., 1 vyučovací hodina (45 minut)
                
                \textbf{Předpoklady:}\\
                    \footnotesize
                    Žáci znají:
                    \vspace{-5mm}
                        \begin{itemize}[noitemsep,topsep=0pt]
                                \item \textbf{posloupnost} jako speciální případ funkce (definičním oborem $D(f)$ je množina \mathbb{N}, příp. její konečná podmnožina), 
                                \item \textbf{(ne)konečná posloupnost},
                                \item \textbf{$n$-tý člen posloupnosti} $a_n$ jako funkční hodnota posloupnosti $f$ v bodě $n$,
                                \item \textbf{symbolický zápis (nekonečné) posloupnosti} (tj. $a_1$, $a_2$,\dots $a_n$),
                                \item \textbf{způsoby zadání posloupnosti} -- vzorcem pro $n$-tý člen nebo rekurentně,
                                \item \textbf{grafické znázornění posloupností} -- graf posloupnosti (jako funkce) nebo znázorněním bodů $a_n$ na číselné ose,
                                \item \textbf{vlastnosti/druhy posloupností} -- (ne)rostoucí, (ne)klesající, (zdola, shora) omezené.
                        \end{itemize}
                \normalsize
        \end{center}
    \end{titlepage}


    \newpage

    % \section{Historie}
    % Úvodní uvedení faktu o znalosti aritmetické posloupnosti již ve starověkém Egyptě,
    % Babylonii a ve staré Číně.

    % Zmínění Fibonacciho posloupnosti -- 

    \section{Příklady z praxe}
    \textbf{Odhadovaný čas:} 20 minut

    Uvedení příkladů, se kterými se mohou žáci setkat v běžném životě. Uvedeny budou
    následující příklady:
    \begin{itemize}
        \item Posloupnost přirozených čísel -- společně se určí $a_1$ a diference.
        \item \textbf{Spoření} (neuvažujeme spořící účet v bance, na začátku období 
              máme již něco naspořeno ($a_1 \neq 0$)) a~pravidelně, každý měsíc, ukládáme
              stejnou částku. \textit{Kolik peněz budeme mát naspořeno za $x$ měsíců?}

              \begin{example}
                  Amálka si v minulém roce zvládla našetřit $50$ Kč. Babička Amálky se na 
                  začátku ledna rozhodla, že jí na začátku každého měsíce dá kapesné 
                  $100$ Kč. Amálka si všechny peníze pečlivě ukládá do pokladničky 
                  a~nic neutratí. Kolik peněz bude mít naspořeno v~polovině června, 
                  tj. po šesté výplatě kapesného?
                  \begin{align*}
                      a_1 &= 50; d = 100; n = 7\\
                      a_2 &= 50 + 100 = 150 \quad \text{tj. naspořená částka z loňska + kapesné za leden}\\
                      a_3 &= 150 + 100 \quad \text{tj. naspořená částka za leden + kapesné za únor}\\
                      \dots\\
                      a_7 &= 50 + (7-1)\cdot100 = \doubleunderline{650} \quad \text{tj. Amálka za 6 měsíců naspoří 650 Kč}
                  \end{align*}
                  \begin{center}
                    \begin{tikzpicture}
                    \begin{axis}[
                        % x=2.5cm,y=2.5cm,
                        xlabel={Měsíce},
                        ylabel={Naspořená částka},
                        xmin=0.2,
                        xmax=8,
                        ymin=-0.2,
                        ymax=750,
                        xtick={1,2,3,4,5,6,7},
                        ytick={0,50,150,250,350,450,550,650},
                        legend pos=north west,
                        ymajorgrids=true,
                        xmajorgrids=true,
                        grid style=dashed,
                        ]
                        %\clip(-1.2,-1.2) rectangle (1.2,1.2);
                        
                        \addplot+[
                            color=blue,
                            mark=*,
                            only marks
                            ]
                            coordinates{
                                (1,50)(2,150)(3,250)(4,350)(5,450)(6,550)(7,650)
                                };
                                % \legend{Naspořená částka}
                            \end{axis}
                        \end{tikzpicture}
                    \end{center}
                \end{example}

        \item \textbf{Pokles teploty s rostoucí nadmořskou výškou.} Teplota klesá v~průměru
              o \SI{0,65}{\celsius} na $100$ výškových metrů, tj. zvýší-li se nadmořská výška o 100~m.~n.~m., 
              ochladí se v~místě v~průměru o \SI{0,65}{\celsius}. \textit{Jaké můžeme očekávat oteplení/ochlazení,
              pokud se dostaneme do vyšší/nižší nadmořské výšky?}

              \begin{example}
                Průměrná nadmořská výška Prahy, odkud jedeme na dovolenou, je 235~m.~n.~m. Chceme jet lyžovat
                do nedaleké Pece pod Sněžkou, jejíž průměrná výška je 769~m.~n.~m. O~kolik se bude lišit 
                teplota oproti Praze, zanedbáme-li všechny ostatní vlivy počasí? 
                
                \begin{align*}
                    235 - 769  &= -534 \quad \text{rozdíl nadmořských výšek v pořadí \uv{start -- cíl}}\\
                    -534 \cdot \dfrac{0,65}{100} &= \doubleunderline{-3,471} \quad \text{teplota se sníží o \SI{3,471}{\celsius}} 
                \end{align*}
            \end{example}
        \item Uvědomění si, že rozdíl každého následujícího a předchozího členu je konstantní. 
    \end{itemize}
    
    \section{Odvození vzorců}
    \textbf{Odhadovaný čas:} 10 minut
    \begin{itemize}
        \item Odvození definičního vzorce pro aritmetickou posloupnost -- (rekurentní zadání)
              $$a_{n+1} = a_n + d.$$ 
        \item Odvození vzorce pro $n$-tý člen aritmetické posloupnosti matematickou indukcí 
              $$a_n = a_1 + (n - 1) \cdot d.$$
    \end{itemize}

    \section{Definice}
    \textbf{Odhadovaný čas:} 5 minut
    \begin{definition}
        \textit{Aritmetická posloupnost} je každá posloupnost určená rekurentně vztahy 
        $a_1 = a$, $a_{n+1} = a_n + d \; \forall n \in \mathbb{N}$ kde $a$, $d$ jsou reálná čísla. 
        Číslo $d$ se nazývá \textit{diference aritmetické posloupnosti}.
    \end{definition}

    Název aritmetické posloupnosti je odvozen od toho, že každý její člen (s výjimkou prvního) je roven
    aritmetickému průměru sousedních členů.\\
    Aritmetická posloupnost s reálnými členy úzce souvisí s lineární funkcí $f:y=ax+b, x\in \mathbb{R}$, 
    kde $a, b \in \mathbb{R}$ jsou daní čísla -- dostaneme ji z této funkce, omezíme-li její definiční 
    obor na množinu \mathbb{N}. 

    \section{Odvození vzorce pro součet}
    \textbf{Odhadovaný čas:} 10 minut\\
        Odvození vzorce pro součet prvních $n$ členů aritmetické posloupnosti matematickou indukcí 
        $$s_n = \dfrac{n}{2} \cdot (a_1 + a_n).$$

        \begin{example}
            Motivace příběhem o C. F. Gaussovi a jeho úloze \uv{Sečti všechna přirozená čísla od $1$ do $100$.}\\
            \begin{gather*}
                (1 + 100) + (2 + 99) + \dots + (100 + 1) = 100 \cdot 101 = 10 100 \\
                \Downarrow\\
                1 + \cdot + 100 = \dfrac{10 100}{2} = \doubleunderline{5 050}        
            \end{gather*}
        \end{example}
            
\end{document}
