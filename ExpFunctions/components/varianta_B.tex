\begin{center}
\large \textbf{Písemná práce:} exponenciální rovnice \textbf{(varianta B)}

\normalsize
\makebox[8cm]{\textsc{Jméno:}\enspace\hrulefill}\qquad
\makebox[3cm]{\textsc{Třída:}\enspace\hrulefill}\qquad
\makebox[4cm]{\textsc{Datum:}\enspace\hrulefill}
\end{center}
\begin{table}[h]
\centering
\begin{tabular}{c|c|c|c|c|c}
    \textbf{Body}   & $10-9$ & $8-7$ & $6-5$ & $4-3$ & $2-0$ \\ \hline
    \textbf{Známka} & $1$     & $2$   & $3 $  & $4$   & $5$
\end{tabular}
\end{table}

\noindent
% First Question
\begin{questions}
    \bracketedpoints
    \question[3] Řešte v $\mathbb{R}$. $$\left(\dfrac{1}{9}\right)^{\frac{x}{x-2}} = 3^{\frac{1}{2}}$$
    \question[3] Řešte v $\mathbb{Z}$. $$\sqrt[6]{6^{5x}} \cdot \sqrt[3]{6^{x+5}} = \sqrt[4]{36}$$
    \question[4] Řešte v $\mathbb{Z}$. $$2\cdot0,5^{x^2+\frac{8}{3}x}=\frac{8}{\sqrt[3]{4}}$$
\end{questions}
\newpage

\begin{center}
\Large\textbf{Vzorové řešení}\normalsize
\begin{enumerate}
    \item Určení podmínek řešitelnosti: $x\neq 2$. \rightnote{1 bod}
      \begin{align*}
        \begin{aligned}
          \left(\dfrac{1}{9}\right)^{\frac{x}{x-2}} &= 3^{\frac{1}{2}}\\
          3^{-2\cdot\frac{x}{x-2}} &= 3^{\frac{1}{2}} & \text{\rightnote{převedení na společný základ (1 bod)}}\\
          -2\cdot\frac{x}{x-2} &= \frac{1}{2}\\
          -4x &= x-2\\
          -5x &= -2\\
          x &= \dfrac{2}{5} & \text{\rightnote{správný výsledek (1 bod)}}
        \end{aligned}
      \end{align*}
    \item Definiční obor rovnice není omezen.
    \begin{align*}
      \begin{aligned}
        \sqrt[6]{6^{5x}} \cdot \sqrt[3]{6^{x+5}} &= \sqrt[4]{36}\\
        6^{\frac{5x}{6}} \cdot 6^{\frac{x+5}{3}} &= 36^{\frac{1}{4}}\\
        6^{\frac{5x}{6}} \cdot 6^{\frac{x+5}{3}} &= 6^{\frac{1}{2}}\\
        6^{\frac{5x}{6} + \frac{x+5}{3}} &= 6^{\frac{1}{2}}&\text{\rightnote{převedení na společný základ (2 body)}}\\
        \frac{5x}{6} + \frac{x+5}{3} &= \frac{1}{2}\\
        5x + 2x + 10 &= 3\\
        7x &= -7\\
        x &= -1&\text{\rightnote{správný výsledek (1 bod)}}
      \end{aligned}
    \end{align*}

    \item Definiční obor není omezen.
    \begin{align*}
      \begin{aligned}
        2\cdot0,5^{x^2+\frac{8}{3}x}&=\frac{8}{\sqrt[3]{4}}\\
        \left(\frac{1}{2}\right)^{x^2+\frac{8}{3}x}&=\frac{2^3}{2^{\frac{2}{3}}}\cdot\frac{1}{2}\\
        \left(\frac{1}{2}\right)^{x^2+\frac{8}{3}x}&=2^{\frac{4}{3}}\\
        \left(\frac{1}{2}\right)^{x^2+\frac{8}{3}x}&=\left(\frac{1}{2}\right)^{-\frac{4}{3}}& \text{\rightnote{převedení na společný základ (2 body)}}\\
        x^2+\frac{8}{3}x +\frac{4}{3} &= 0\\
        3x^2+8x +4 &= 0\\
        x_{1,2}&=\frac{-8\pm\sqrt{8^2-4\cdot 3\cdot7}}{2\cdot 3} & \\
        x_1=-2 &\land \cancel{x_2=-\frac{2}{3}} & \text{\rightnote{správný výsledek a vyřazení $x_2$ (2 body)}}
      \end{aligned}
    \end{align*}
\end{enumerate}
\end{center}