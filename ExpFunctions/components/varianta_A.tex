\begin{center}
\large \textbf{Písemná práce:} exponenciální rovnice \textbf{(varianta A)}

\normalsize
\makebox[8cm]{\textsc{Jméno:}\enspace\hrulefill}\qquad
\makebox[3cm]{\textsc{Třída:}\enspace\hrulefill}\qquad
\makebox[4cm]{\textsc{Datum:}\enspace\hrulefill}
\end{center}
\begin{table}[h]
\centering
\begin{tabular}{c|c|c|c|c|c}
    \textbf{Body}   & $10-9$ & $8-7$ & $6-5$ & $4-3$ & $2-0$ \\ \hline
    \textbf{Známka} & $1$     & $2$   & $3 $  & $4$   & $5$
\end{tabular}
\end{table}

\begin{questions}
    \question[3] Řešte v $\R$ rovnici $$4^{\frac{1-x}{1+x}}=\left(\frac{1}{2}\right)^{\frac{1}{3}}$$
    \question[3] Řešte v $\Z$ rovnici $$\sqrt[4]{4^x}\cdot \sqrt[3]{2^{x-3}}=\sqrt[6]{16}$$.
    \question[4] Řešte v $\Z$ rovnici $$\sqrt[5]{3^{8x^2}}\cdot\left(\frac{1}{9}\right)^{\frac{x^2}{2}}\cdot 3^x=\sqrt[5]{27^4}$$.
\end{questions}

\newpage

\begin{center}
\Large\textbf{Vzorové řešení}\normalsize
\begin{enumerate}
    \item Určení podmínek řešitelnosti: $x\neq -1$.\rightnote{1 bod}
    \begin{align*}
        \begin{aligned}
            4^{\frac{1-x}{1+x}}&=\left(\frac{1}{2}\right)^{\frac{1}{3}} & \\
            2^{\frac{2-2x}{1+x}}&=2^{-\frac{1}{3}} & \text{\rightnote{převedení na společný základ (1 bod)}}\\
            \frac{2-2x}{1+x}&=-\frac{1}{3} & \\
            6-6x&=-1-x & \\
            7&=5x & \\
            x&=\frac{7}{5} & \text{\rightnote{správný výsledek (1 bod)}}
        \end{aligned}
    \end{align*}
    \item Definiční obor rovnice není omezen.
    \begin{align*}
        \begin{aligned}
            \sqrt[4]{4^x}\cdot \sqrt[3]{2^{x-3}}&=\sqrt[6]{16} & \\
            2^{\frac{x}{2}}\cdot 2^{\frac{x-3}{3}}&=2^{\frac{2}{3}} & \text{\rightnote{převedení na společný základ (2 body)}}\\
            \frac{x}{2}+\frac{x-3}{3}&=\frac{2}{3} & \\
            3x+2x-6&=4 & \\
            5x&=10 & \\
            x&=2 & \text{\rightnote{správný výsledek (1 bod)}}
        \end{aligned}
    \end{align*}
    \item Definiční obor rovnice není omezen.
    \begin{align*}
        \begin{aligned}
            \sqrt[5]{3^{8x^2}}\cdot\left(\frac{1}{9}\right)^{\frac{x^2}{2}}\cdot 3^x&=\sqrt[5]{27^4} & \\
            3^{\frac{8}{5}}\cdot 3^{-x^2}\cdot 3^x&=3^{\frac{12}{5}}& \text{\rightnote{převedení na společný základ (2 body)}}\\
            \frac{8}{5}x^2-x^2+x&=\frac{12}{5} & \\
            3x^2+5x-12&=0 & \\
            x_{1,2}&=\frac{-5\pm\sqrt{5^2-4\cdot 3\cdot (-12)}}{2\cdot 3} & \\
            x_1=-3 &\land \cancel{x_2=-\frac{4}{3}} & \text{\rightnote{správný výsledek a vyřazení $x_2$ (2 body)}} \\
        \end{aligned}
    \end{align*}
\end{enumerate}
\end{center}
