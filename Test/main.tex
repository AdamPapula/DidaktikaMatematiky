\documentclass[12pt,a4paper]{exam}

% Packages
\usepackage{defpackages}
\usepackage{mathcmd}
\usepackage{optionalcmd}
\usepackage{pgfplots}
\pgfplotsset{compat=1.15}
\usepackage{mathrsfs}

% Assets
\def\hmargin{18mm}
\def\vmargin{25mm}

\geometry{
	top=\vmargin,
	bottom=\vmargin,
	left=\hmargin,
	right=\hmargin
}
\setlength{\parindent}{0pt}
\setlength{\parskip}{\baselineskip}
\setlength{\dottedlinefillheight}{1cm}


% Title page info
\def\maintitle{Pedagogická propedeutika}
\def\subtitle{Exponenciální funkce (test)}
\def\authorname{Adam Papula, David Weber}
\def\currentdate{\today}

% Language adjustment
\chpword{b.}
\pointpoints{b.}{b.}
\bonuspointpoints{b.}{b.}

\begin{document}

    % First Question
    \pagestyle{empty}
    \begin{questions}

        \question[2] Teoretická část.
        \begin{parts}
            \part[1] Napište definici exponenciální funkce.
            \fillwithdottedlines{2cm}
            \part[1] Stručně vysvětlete, proč klademe na hodnotu základu omezení.
            \fillwithdottedlines{2cm}
        \end{parts}
        \question[2] Vyberte funkční předpis odpovídající grafu funkce $f$ níže.\\
        \begin{center}
            \begin{tikzpicture}[>=latex]
                \begin{axis}[
                x=1.5cm,y=1.5cm,
                axis x line=center,
                axis y line=center,
                xtick={-3,-2,...,5},
                ytick={-2,-1,...,3},
                xlabel={$x$},
                ylabel={$y$},
                xlabel style={below right},
                ylabel style={above left},
                xmin=-3,
                xmax=3.5,
                ymin=-2,
                ymax=3.5,
                anchor=center]
                    \addplot [mark=none,domain=-2.9:3.4] {(1/2)^(x+1)-1} node[pos=1] (endofplotsquare) {};
                    \node [above,color=black] at (endofplotsquare) {$f$};
                    \addplot [dotted,mark=none] {-1};
                \end{axis}
            \end{tikzpicture}
        \end{center}
        \begin{checkboxes}
            \choice $\displaystyle f: y=\left(\frac{1}{2}\right)^{x-1}+1$
            \CorrectChoice $\displaystyle f: y=\left(\frac{1}{2}\right)^{x+1}-1$
            \choice $\displaystyle f: y=2^{x-1}+1$
            \choice $\displaystyle f: y=2^{x+1}-1$
            \choice Žádný z uvedených.
        \end{checkboxes}

        \newpage

        \question[6] Mějme reálnou funkci $\displaystyle g: y=-2^{x-2}+\frac{1}{2}$, kde $\displaystyle D_g=\left(-4,\frac{1}{2}\right\rangle$. Ke každé části uveďte postup řešení. 
        \begin{parts}
            \part[4] Nakreslete graf funkce $g$.
            \part[1] Určete obor hodnot $H_g$.
            \part[1] Určete průsečík grafu $g$ s osou $y$.
        \end{parts}

        \newpage

        \bonusquestion[4] Mějme reálnou funkci $\displaystyle h: y=\left(\frac{a+1}{a^2-1}\right)^x$. Určete, pro jaké hodnoty parametru $a\in\R$ je funkce $h$ klesající. Uveďte celý postup řešení.

    \end{questions}

\end{document}
