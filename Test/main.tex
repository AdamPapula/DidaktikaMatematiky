\documentclass[12pt,a4paper,addpoints]{exam}

% Packages
\usepackage{defpackages}
\usepackage{mathcmd}
\usepackage{optionalcmd}
\usepackage{pgfplots}
\pgfplotsset{compat=1.15}
\usepackage{mathrsfs}

% Assets
\def\hmargin{18mm}
\def\vmargin{25mm}

\geometry{
	top=\vmargin,
	bottom=\vmargin,
	left=\hmargin,
	right=\hmargin
}
\setlength{\parindent}{0pt}
\setlength{\parskip}{\baselineskip}
\setlength{\dottedlinefillheight}{1cm}


% Title page info
\def\maintitle{Pedagogická propedeutika}
\def\subtitle{Exponenciální funkce (test)}
\def\authorname{Adam Papula, David Weber}
\def\currentdate{\today}

% Language adjustment
\chpword{b.}
\pointpoints{b.}{b.}
\bonuspointpoints{b. -- \textbf{bonusová úloha}}{b. -- \textbf{bonusová úloha}}



%Checkbox adjustment
\checkboxchar{$\Box$}
\checkedchar{$\blacksquare$}

\begin{document}
    \begin{center}
      \large \textbf{Písemná práce:} exponenciální funkce

      \normalsize
      \makebox[8cm]{\textsc{Jméno:}\enspace\hrulefill}\qquad
      \makebox[3cm]{\textsc{Třída:}\enspace\hrulefill}\qquad
      \makebox[4cm]{\textsc{Datum:}\enspace\hrulefill}
    \end{center}
    \begin{table}[h]
      \centering
      \begin{tabular}{c|c|c|c|c|c}
        \textbf{Body}   & $< 9$ & $8-7$ & $6-5$ & $4-3$ & $2-0$ \\ \hline
        \textbf{Známka} & $1$     & $2$   & $3 $  & $4$   & $5$
      \end{tabular}
    \end{table}

    \noindent
    % First Question
    \begin{questions}
        \pagestyle{empty}
        \bracketedpoints

        \question[1] Napište definici exponenciální funkce.
                    \fillwithdottedlines{3\dottedlinefillheight}
        \question[1] Stručně vysvětlete, proč klademe na hodnotu základu omezení.
                     \fillwithdottedlines{3\dottedlinefillheight}
        \question[2] Vyberte funkční předpis odpovídající grafu funkce $f$ níže.\\
          \begin{center}
              \begin{tikzpicture}[>=latex]
                  \begin{axis}[
                  x=1cm,y=1cm,
                  axis x line=center,
                  axis y line=center,
                  xtick={-3,-2,...,5},
                  ytick={-2,-1,...,3},
                  minor tick num=1,
                  xlabel={$x$},
                  ylabel={$y$},
                  xlabel style={below right},
                  ylabel style={above left},
                  xmin=-2.9,
                  xmax=3.4,
                  ymin=-1.4,
                  ymax=3.4,
                  anchor=center]
                      \addplot [domain=-2.8:3.3] {(1/2)^(x+1)-1} node[pos=1] (endofplotsquare) {};
                      \node [above,color=black] at (endofplotsquare) {$f$};
                      \node[label={180:{$-\frac{1}{2}$}},inner sep=2pt] at (axis cs:0,-0.5) {};
                      \addplot [dotted,mark=none] {-1};
                      \addplot [dotted,mark=none,domain=-2:0] {1};
                      \addplot [dotted,mark=none,domain=0:1] coordinates {(-2,0)(-2,1)};
                      \node[circle,fill,inner sep=2pt,scale=0.5] at (axis cs:-2,1) {};
                  \end{axis}
              \end{tikzpicture}
          \end{center}

          \begin{checkboxes}
            \choice $f: y=\left(\tfrac{1}{2}\right)^{x-1}+1$
            \CorrectChoice $f: y=\left(\tfrac{1}{2}\right)^{x+1}-1$
            \choice $f: y=2^{x-1}+1$
            \choice $f: y=2^{x+1}-1$
            \choice Žádný z uvedených.
          \end{checkboxes}

        \newpage

        \question[6] Mějme reálnou funkci $g: y=-2^{x-2}+\tfrac{1}{2}$, kde $D_g=\left(-4,3\right\rangle$. %Ke  každé části uveďte postup řešení.
          \begin{parts}
              \renewcommand{\thepartno}{\Alph{partno}}
              \part[4] Nakreslete graf funkce $g$.
              \part[1] Určete obor hodnot $H_g$.
              \part[1] Určete průsečík grafu $g$ s osou $x$ a $y$.
          \end{parts}
          \begin{center}
              \begin{tikzpicture}[>=latex]
                  \begin{axis}[
                  x=1.5cm,y=1.5cm,
                  axis x line=center,
                  axis y line=center,
                  xtick={-5,-4,...,5},
                  ytick={-5,-4,...,5},
                  minor tick num=1,
                  grid=both,
                  grid style={line width=.1pt, draw=gray!30},
                  major grid style={line width=.2pt,draw=gray!50},
                  xlabel={$x$},
                  ylabel={$y$},
                  xlabel style={below right},
                  ylabel style={above left},
                  xmin=-4.9,
                  xmax=4.9,
                  ymin=-4.9,
                  ymax=4.9,
                  anchor=center]
                  \end{axis}
              \end{tikzpicture}
          \end{center}
        \newpage

        \bonusquestion[4] Mějme reálnou funkci $h: y=\left(\tfrac{a+1}{a^2-1}\right)^x$. Určete, pro jaké hodnoty parametru $a\in\R$ je funkce $h$ klesající. Uveďte celý postup řešení.
        \newpage
    \end{questions}

    \begin{center}
      \Large\textbf{Vzorové řešení}\normalsize
      \begin{enumerate}
        \item Nechť $a \in \mathbb{R}^+ \setminus \{ 1 \}$. Exponenciální funkcí i základu $a$ se nazývá funkce $f$ daná rovnicí $y=a^x$,     jejím definičním oborem je $D(f) = \mathbb{R}$.

        \item Pro $a=1$ je $y=1^x = 1$ pro každé $x \in \mathbb{R}$, tj. funkce je konstantní, proto tento případ u exponenciální funkce vylučujeme.
        Pokud by základ byl záporný, např. mějme funkci $f(x) = (-2)^x$. Když za $x$ dosadíme $\tfrac{1}{2}$, dostaneme $y=(-2)^{\tfrac{1}{2}} = \sqrt{-2}$. My ale víme, že odmocnina ze záporného čísla v $\mathbb{R}$ neexistuje. Funkce by tak nebyla definovaná na celém $\mathbb{R}$.

        \item $f: y=\left(\tfrac{1}{2}\right)^{x+1}-1$
        \item Graf:\\
        \begin{center}
            \begin{tikzpicture}[>=latex]
                \begin{axis}[
                x=1cm,y=1cm,
                axis x line=center,
                axis y line=center,
                xtick={-4,-3,...,3},
                ytick={-2,-1,...,3},
                minor tick num=1,
                xlabel={$x$},
                ylabel={$y$},
                xlabel style={below right},
                ylabel style={above left},
                xmin=-4.3,
                xmax=3.7,
                ymin=-2.4,
                ymax=1.4,
                anchor=center]
                    \addplot [domain=-4:3] {-2^(x-2)+(1/2)} node[pos=1] (endofplotsquare) {};
                    \node [above,color=black] at (endofplotsquare) {$g$};
                    \addplot [dotted,mark=none] {0.5};
                    \coordinate (begin) at (axis cs:-4,31/64);
                    \coordinate (end) at (axis cs:3,-3/2);
                \end{axis}
                \filldraw[draw=black,fill=white]  (begin) circle(0.7mm);
                \filldraw[fill=black,draw=black] (end) circle(0.7mm);

            \end{tikzpicture}
        \end{center}

        $$H_g = \left(-\tfrac{31}{64},\tfrac{-3}{2}\right\rangle$$
        $$P_x = \left[1,0\right]$$
        $$P_y = \left[0,\tfrac{1}{4}\right]$$
        \item  \todo{$h: y=\left(\tfrac{a+1}{a^2-1}\right)^x$}
      \end{enumerate}

    \end{center}



\end{document}
