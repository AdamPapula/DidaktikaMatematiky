\documentclass[12pt,a4paper,addpoints]{article}

\pagestyle{empty}

% Packages
\usepackage{../Styles/defpackages}
\usepackage{../Styles/mathcmd}
\usepackage{../Styles/optionalcmd}
\usepackage{pgfplots}
\pgfplotsset{compat=1.15}
\usepackage{mathrsfs}

% Assets
\def\hmargin{18mm}
\def\vmargin{25mm}
\newcommand{\rightnote}[1]{\hspace*{\fill} $\triangleleft$ \textit{#1}}

\geometry{
	top=\vmargin,
	bottom=\vmargin,
	left=\hmargin,
	right=\hmargin
}
\setlength{\parindent}{0pt}
\setlength{\parskip}{\baselineskip}
\setlength{\dottedlinefillheight}{1cm}


% Title page info
\def\maintitle{Didaktika matematiky}
\def\subtitle{Aplikační úloha}
\def\authorname{Kateřina Novotná, Adam Papula}


\begin{document}
    \begin{titlepage}
        \begin{center}
            \Large\textbf{{\maintitle}}

            \normalsize
            \vspace{0.5cm}
                \subtitle
            \vspace{1.5cm}
            
            \textbf{\authorname}
            \vspace{1.5cm}

            \today
            \vfill
            
            \raggedright
                \textbf{Téma:} Přirozený logaritmus.
                
                \textbf{Cílová skupina žáků:} 2. ročník SŠ.

                \textbf{Cíl úlohy:} 
                \footnotesize
                    \vspace{-5mm}
                    \begin{itemize}
                        \item Žák aplikuje přirozený logaritmus na úlohu ze života.
                        \item Žák správně pracuje se vzorcem (tzn. dosazení,\dots).
                        \item Žák správně pracuje s neznámou ve vzorci.
                        \item Žák vyjádří neznámou v exponentu ze vzorce.
                    \end{itemize}
                \normalsize
                \textbf{Předpoklady:}\\
                \footnotesize
                Žáci znají:
                \vspace{-5mm}
                    \begin{itemize}[noitemsep,topsep=0pt]
                            \item přirozený logaritmus,
                            \item práce se vzorcem, dosazení do vzorce,
                            \item vyjádření neznámé ze vzorce,
                            \item exponenciální funkce.
                    \end{itemize}
            \normalsize
        \end{center}
    \end{titlepage}


    \newpage

    \section{Zadání}
    Ideálně upečený cheesecake čerstvě vyndaný z trouby má teplotou \SI{74}{\celsius}. Umístíme jej do ledničky, ve které je teplota 
    nastavena na \SI{1,7}{\celsius}. Po 10~minutách cheesecake změříme a~zjistíme, že jeho vnitřní teplota klesla na \SI{65,6}{\celsius}.
    Nejvhodnější teplota na servírování je pokojová teplota. Zákazník požaduje přesnou teplotu  \SI{21,1}{\celsius}. Kdy nejpozději 
    musíme začít péct, víme-li, že se cheesecake peče $75$ minut?
   
    \subsection{Nápověda}
    Teplota ($T$) objektu v prostředí se vzduchem o teplotě $T_v$ se bude chovat podle vzorce
    $$T(t) = D_0 \cdot \textit{e}^{-tk} + T_v$$
    Kde 
    \begin{itemize}
        \item $t$ je čas,
        \item $D_0$ je rozdíl mezi počáteční teplotou objektu a prostředí (v našem případě teplota cheesecake a lednice),
        \item $k$ je konstanta kontinuální rychlosti ochlazování objektu.
    \end{itemize}

    Nejprve je potřeba zjistit konstantu dopočítáním z již známých údajů. Následně tuto konstantu použít pro výpočet požadovaných údajů. 

    \newpage
    \section{Řešení}
    Víme:
    $$T(t) = D_0 \cdot \textit{e}^{-tk} + 1,7$$
    \textit{Poznámka: Zde je nutné ze zadání správně vyčíst jednotlivé údaje.}
    
    Dopočítání konstanty, \begin{itemize}
        \item $T(0) = 74$, tedy (nic nám nedá, pouze ověření):
        $$74 = (74 - 1,7)\cdot \textit{e}^{-0k} + 1,7$$ 
        \item $T(10) = 65,6$ a tedy:
        \begin{align*}
            65,6 &= (74 - 1,7)\cdot \textit{e}^{-10k} + 1,7\\
            63,9 &= 72,3 \cdot \textit{e}^{-10k}\\
            \frac{63,9}{72,3} &= \textit{e}^{-10k}\\
            \ln{\frac{63,9}{72,3}} &= -10k\\
            k &= \frac{\ln{\frac{63,9}{72,3}}}{-10}
        \end{align*}
    \end{itemize}
    \textit{Poznámka: Žáci si musí uvědomit, že nechtějí vyjádřit $e$, ale proměnnou $k$.}

    Nyní, když známe konstantu, můžeme ji doplnit do vzorce pro ochlazování objektu.
    $$T(t) = 72,3 \cdot \textit{e}^{t \cdot (\frac{1}{10} \cdot \ln{\frac{63,9}{72,3}})} + 1,7$$
    Dosazením $21,1$ za $T(t)$ a následným dopočítáním proměnné $t$ zjistíme, jak dlouho cheesecake bude chladnout.
    \textit{Poznámka: Méně pečliví žáci mohou mít problém se zápisem exponentu u $e$, může nastat problém se znaménkem minus v exponentu.}

    \textit{Poznámka: Zde si musí uvědomit, že $T(t)$ je funkce závislosti teploty na čase, $t$ je čas ke kterému se chtějí dopracovat, $T(t)$ znají.}
    \begin{align*}
        21,1 &= 72,3 \cdot \textit{e}^{t \cdot (\frac{1}{10} \cdot \ln{\frac{63,9}{72,3}})} + 1,7\\
        19,4 &= 72,3 \cdot \textit{e}^{t \cdot (\frac{1}{10} \cdot \ln{\frac{63,9}{72,3}})}\\
        \frac{19,4}{72,3} &= \textit{e}^{t \cdot (\frac{1}{10} \cdot \ln{\frac{63,9}{72,3}})}\\
        \ln{\frac{19,4}{72,3}} &= t \cdot (\frac{1}{10} \cdot \ln{\frac{63,9}{72,3}})\\
        t &= \frac{\ln{\frac{19,4}{72,3}}}{\frac{1}{10} \cdot \ln{\frac{63,9}{72,3}}}\\
        t &\approx 106,5
    \end{align*}


    \textit{Poznámka: Výsledek $106,5$ minut není konečný výsledek, je potřeba ještě připočítat čas pečení.}\\
    \textit{Poznámka: Může nastat problém se správným určením jednotky u výsledku.}

    Chladnutí dortu bude trvat $106,5$ minuty, samotné pečení zabere dalších $75$ minut. Nejpozději musíme začít $181,5$ ($3$ hodiny a $1,5$ minuty) minuty před příchodem zákazníka.



\end{document}
