\documentclass[12pt,a4paper,addpoints]{article}

\pagestyle{empty}

% Packages
\usepackage{../Styles/defpackages}
\usepackage{../Styles/mathcmd}
\usepackage{../Styles/optionalcmd}
\usepackage{pgfplots}
\pgfplotsset{compat=1.15}
\usepackage{mathrsfs}

% Assets
\def\hmargin{18mm}
\def\vmargin{25mm}
\newcommand{\rightnote}[1]{\hspace*{\fill} $\triangleleft$ \textit{#1}}

\geometry{
	top=\vmargin,
	bottom=\vmargin,
	left=\hmargin,
	right=\hmargin
}
\setlength{\parindent}{0pt}
\setlength{\parskip}{\baselineskip}
\setlength{\dottedlinefillheight}{1cm}


% Title page info
\def\maintitle{Didaktika matematiky}
\def\subtitle{Aplikační úloha}
\def\authorname{Kateřina Novotná, Adam Papula}


\begin{document}
    \section{Zadání}
    Ideálně upečený cheesecake čerstvě vyndaný z trouby má teplotou \SI{74}{\celsius}. Umístíme jej do ledničky, ve které je teplota 
    nastavena na \SI{1,7}{\celsius}. Po 10~minutách cheesecake změříme a~zjistíme, že jeho vnitřní teplota klesla na \SI{65,6}{\celsius}.
    Nejvhodnější teplota na servírování je pokojová teplota. Zákazník požaduje přesnou teplotu  \SI{21,1}{\celsius}. Kdy nejpozději 
    musíme začít péct, víme-li, že se cheesecake peče $75$ minut?
   
    \subsection{Nápověda}
    Teplota ($T$) objektu v prostředí se vzduchem o teplotě $T_v$ se bude chovat podle vzorce
    $$T(t) = D_0 \cdot \textit{e}^{-tk} + T_v$$
    Kde 
    \begin{itemize}
        \item $t$ je čas,
        \item $D_0$ je rozdíl mezi počáteční teplotou objektu a prostředí (v našem případě teplota cheesecake a lednice),
        \item $k$ je konstanta kontinuální rychlosti ochlazování objektu.
    \end{itemize}

    Nejprve je potřeba zjistit konstantu dopočítáním z již známých údajů. Následně tuto konstantu použít pro výpočet požadovaných údajů. 

    \newpage


\end{document}
